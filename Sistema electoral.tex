\documentclass[11pt]{article}
\usepackage [utf8]{inputenc}
\usepackage [spanish]{babel}
\usepackage{multicol}
\usepackage{parskip}
\usepackage[margin=3cm]{geometry}
\usepackage{enumerate}
\usepackage{graphicx}
\usepackage{mathtools}
\usepackage{setspace}
\usepackage{pgfplots}
\usepackage{pgf-pie}
\usepackage{amsfonts}
\usepackage{xcolor}
\usepackage{framed}

\usepackage{lipsum}

%---- Fuentes

\usepackage{fontspec}
\setsansfont{Montserrat}[
	Extension		= .ttf ,
	UprightFont		= *-Regular ,
	BoldFont		= *-Bold]
\newfontfamily\headingfont{Montserrat}[
	Extension		= .ttf ,
	UprightFont		= *-Regular ,
	BoldFont		= *-Bold]
\setmainfont{Lora}[
	Extension		= .ttf ,
	UprightFont		= *-Regular ,
	BoldFont		= *-Bold ,
	ItalicFont		= *-Italic ,
	BoldItalicFont	= *-BoldItalic]
\newfontfamily\inconsolata{Inconsolata}[
	Extension		= .ttf ,
	UprightFont		= *-Regular ,
	BoldFont		= *-Bold]

\newcommand\console[1]{{\inconsolata #1}}

%----- Encabezados

\usepackage{titlesec}
\usepackage{titling}

\renewcommand{\maketitlehooka}{\headingfont}

\titleformat*{\section}{\huge\bfseries\sffamily}
\titleformat*{\subsection}{\Large\bfseries\sffamily}
\titleformat*{\subsubsection}{\large\bfseries\sffamily}
\titleformat*{\paragraph}{\large\bfseries\sffamily}

%----- Colores

\definecolor{pp}{HTML}{00A3DF}
\definecolor{psoe}{HTML}{EE1C25}
\definecolor{podemos}{HTML}{672F6C}
\definecolor{ciudadanos}{HTML}{EB6109}

\begin{document}
	\title{
		Sistema electoral español\\
		\vspace*{1\baselineskip}
		\large 1ºDGIIM \\
		Métodos Numéricos 
		\vspace*{1\baselineskip}
	} 
	
	\date{31-03-16}
	\author{Antonio Coín Castro \and José María Martín Luque \and Miguel Lentisco Ballesteros}
	
	%\maketitle
	
	\fontsize{72}{60}{
		\headingfont{\raggedright{\textbf{Sistema}}} \\\\
		\headingfont{\raggedright{\textbf{Electoral}}} \\\\
		\headingfont{\raggedright{\textbf{Español}}} \\\\\\\\
	}
	
	\vfill
	
	\fontsize{20}{28}{
		\inconsolata{
			\raggedright{Antonio Coín Castro} \\
			\raggedright{Miguel Lentisco Ballesteros} \\
			\raggedright{José María Martín Luque} \\
		}
	}
	\thispagestyle{empty}
	\fontsize{11}{14}
	
	\newpage
	
	\tableofcontents
	
	\newpage
	
	\section*{Preámbulo.}
	\addcontentsline{toc}{section}{Preámbulo.}
	
	Todas las democracias tienen algo en común: \textit{la soberanía nacional reside en el pueblo}. Ahí se acaban las coincidencias: hay muchas formas de implementar la democracia. La más extendida es la democracia representativa: el pueblo elige a representantes que tomarán decisiones por ellos en los órganos de poder. España es un país con una democracia representativa. Los ciudadanos españoles eligen mediante el voto a sus representantes en el Congreso de los Diputados y el Senado, sedes del poder legislativo. Para transformar \textit{votos} en \textit{escaños} (así es como se denominan cada uno de los asientos que ocupan los representantes políticos en las cámaras parlamentarias) existen diversos métodos o sistemas. Explicar el funcionamiento de nuestro sistema es el tema que nos ocupa en este trabajo.
	
	El sistema electoral español para las elecciones al Congreso de los Diputados es un sistema sencillo de comprender, al menos comparado con el de otros países como Alemania o Reino Unido. El objetivo de este documento no es sólo explicar el funcionamiento y en qué leyes se fundamenta el sistema electoral de nuestro país. También pretende mostrar los problemas que presenta, y exponer las principales críticas que recibe. 
	
	Para exponer mejor estas ideas este documento no sólo recoge información sobre nuestro sistema electoral. Ver el funcionamiento de otros sistemas ayuda a comparar ventajas y deficiencias. Por este motivo, en este texto también se comentan brevemente los sistemas vigentes en Alemania y Reino Unido (países que ya fueron mencionados anteriormente) así otros métodos de reparto de escaños.
	
	Sin duda, apoyarse en datos reales es una de las mejores formas de presentar y explicar información. Este documento no se limita a explicar los fundamentos teóricos de los sistemas electorales analizados, sino que también recoge resultados reales de elecciones. Pero lo más interesante es utilizar estos datos y realizar simulaciones con ellos empleando distintos métodos. El uso de simulaciones permite apreciar de forma mucho más fácil las diferencias entre los sistemas electorales explicados teóricamente.
	
	Una vez finalizada la lectura de este escrito se espera que el lector haya adquirido el conocimiento suficiente como para comprender el funcionamiento de nuestro sistema electoral y cómo se compara con otros alternativos.
		
	\newpage
	
	\section{Sistema electoral español.}
	
	Actualmente el sistema electoral español está sustentado tanto en la Constitución de 1978 como en la Ley Orgánica 5/1985, de 19 de junio, del Régimen Electoral General. En este documento se recoge el sistema electoral para la elección de los diputados del Congreso. El Congreso de los diputados es la sede del Poder Legislativo. Está formado por trescientos cincuenta Diputados, como recoge el artículo 162 de la Ley Electoral. La elección de los miembros de esta cámara se hace por sufragio universal entre los ciudadanos españoles mayores de 18 años.
	
	Los electores votan \textit{listas electorales}, una relación de aspirantes a obtener un escaño que se presentan bajo un mismo partido o coalición. Estas listas deben ser presentadas a nivel provincial. De esta forma, a cada provincia le corresponde un número concreto de diputados. Se dice que las provincias constituyen la circunscripción electoral. Por este motivo, los partidos presentan a sus candidatos en las distintas circunscripciones para intentar conseguir el mayor número de diputados posible.
	
	La Constitución sienta las bases del sistema electoral, pero es una ley la que tendrá que regularlo por completo. Básicamente se limita a mencionar algunos aspectos básicos que debe cumplir dicha ley, como el número de diputados o la elección de éstos. \\
	
	\console{
		\begin{leftbar}
			\textbf{CONSTITUCIÓN ESPAÑOLA.} \\
			\textbf{Artículo 68.}
			
			\begin{enumerate}
			\item El Congreso se compone de un mínimo de 300 y un máximo de 400 Diputados, elegidos por sufragio universal, libre, igual, directo y secreto, en los términos que establezca la ley.
			\item La circunscripción electoral es la provincia. Las poblaciones de Ceuta y Melilla estarán representadas cada una de ellas por un Diputado. La ley distribuirá el número total de Diputados, asignando una representación mínima inicial a cada circunscripción y distribuyendo los demás en proporción a la población.
			\item La elección se verificará en cada circunscripción atendiendo a criterios de representación proporcional. \\
		\end{enumerate}	
		\end{leftbar}
	}
	
	La ley que cumple esta función es la Ley del Régimen Electoral General. En ella se acaban de detallar los procedimientos brevemente esbozados en la Constitución. Da cifras concretas del número de parlamentarios y especifica el método a seguir para traducir votos a escaños así como para calcular el número de escaños de cada circunscripción. Así, los artículos 161, 162 y 163 de esta ley detallan en profundidad los puntos 1, 2 y 3 del artículo 68 de la Constitución. \\
	
	\console{
	\begin{leftbar}
		\textbf{LEY DEL RÉGIMEN ELECTORAL GENERAL.} \\
		\textbf{Artículo 161.}
		
		\begin{enumerate}
			\item Para la elección de Diputados y Senadores, cada provincia constituirá una circunscripción electoral. Asimismo, las ciudades de Ceuta y Melilla serán consideradas, cada una de ellas, como circunscripciones electorales.
		\end{enumerate}
		
		\textbf{Artículo 162.}
		
		\begin{enumerate}
			\item El Congreso está formado por trescientos cincuenta Diputados.
			\item A cada provincia le corresponde un mínimo inicial de dos Diputados. Las poblaciones de Ceuta y Melilla están representadas cada una de ellas por un Diputado.
			\item Los doscientos cuarenta y ocho Diputados restantes se distribuyen entre las provincias en proporción a su población, conforme al siguiente procedimiento:
				\begin{enumerate}[a)]
					\item Se obtiene una cuota de reparto resultante de dividir por doscientos cuarenta y ocho la cifra total de la población de derecho de las provincias peninsulares e insulares.
					\item Se adjudican a cada provincia tantos Diputados como resulten, en números enteros, de dividir la población de derecho provincial por la cuota de reparto.
					\item Los Diputados restantes se distribuyen asignando uno a cada una de las provincias cuyo cociente, obtenido conforme al apartado anterior, tenga una fracción decimal mayor.
				\end{enumerate}
		\end{enumerate}
		
		\textbf{Artículo 163}.
	
		\begin{enumerate}
			\item La atribución de los escaños en función de los resultados del escrutinio se realiza conforme a las siguientes reglas:
			\begin{enumerate}[a)]
					\item No se tienen en cuenta aquellas candidaturas que no hubieran obtenido, al menos, el 3 por 100 de los votos válidos emitidos en la circunscripción.
					\item Se ordenan de mayor a menor, en una columna, las cifras de votos obtenidos por las restantes candidaturas.
					\item Se divide el número de votos obtenidos por cada candidatura por 1, 2, 3, etcétera, hasta un número igual al de escaños correspondientes a la circunscripción, formándose un cuadro similar al que aparece en el ejemplo práctico. Los escaños se atribuyen a las candidaturas que obtengan los cocientes mayores en el cuadro, atendiendo a un orden decreciente.
				\end{enumerate}
		\end{enumerate}

	\end{leftbar}
	}
	
		
	\subsection{Cómo se reparten los escaños.}
	\subsection{Redondeo.}
	\subsection{Críticas al sistema electoral actual.}
	
	\paragraph{Falta de proporcionalidad.}
	
	La Constitución dice en su artículo número 68 que "la elección se verificará \textbf{en cada circunscripción} atendiendo a criterios de representación proporcional". Es decir, nuestro sistema electoral \textbf{no es proporcional} en el conjunto del Estado, pero sí que lo es en la \textbf{provincia}. O lo que es lo mismo, la propia Constitución no asegura que la distribución de escaños a nivel estatal se corresponda de forma proporcional al cómputo total de los votos.
	
	\paragraph{Circunscripción provincial.}
	
	Se suele culpar a la Ley D'Hondt de la falta de proporcionalidad en los resultados Estatales. Sin embargo, la principal causante de este problema (que como se ha comentado en el punto anterior viene implícitamente contemplado en la Constitución) es la circunscripción provincial. Muchos politólogos argumentan que realmente en España hay tres elecciones, con tres sistemas \textit{distintos} atendiendo al número de diputados a elegir por circunscripción. Las circunscripciones que reparten de 1 a 6 escaños son sistemas mayoritarios, el partido que más votos recibe puede acaparar gran parte de los escaños (los datos de todas elecciones nos dan una barrera media efectiva del 15,2\%). Las circunscripciones que reparten entre 6 y 9 escaños se asemejan a un sistema intermedio entre mayoritario y proporcional (la barrera efectiva media aquí es del 9,8\%). Finalmente, las provincias que reparten más de 10 escaños podemos decir que son puramente proporcionales y la barrera efectiva media se acerca a la impuesta por la Ley Electoral, es del 3,1\%.
	
	\paragraph{Listas cerradas.}
	
	Las listas abiertas o desbloqueadas propician el acercamiento los candidatos a los electores, ya que estos deben hacer campaña individualmente para intentar obtener mejores resultados que el resto de candidatos (y obtener el escaño).
	
	
	
	\newpage
	
	\section{Análisis de resultados electorales.}
	\subsection{Elecciones del año 2004.}
	
	Las elecciones generales del año 2004 tuvieron lugar el domingo 14 de marzo. Los españoles acudieron a las urnas a votar tan sólo tres días después de los atentados del 11M, en los que murieron 193 personas. A pesar de que todas las encuestas apuntaban a una victoria del PP, finalmente las elecciones fueron ganadas por el PSOE, lo que permitió a José Luis Rodríguez Zapatero convertirse en presidente del Gobierno.
	
	La composición del Congreso quedó como sigue:
	
	\console{
		\begin{center}
			\begin{tikzpicture}
				\begin{scope}[yscale=1,xscale=-1]
	 		  		\pie[pos={6,0}, sum=700, after number=, radius=4, color={psoe, pp, gray }, text=legend]{164/PSOE, 148/PP, 38/Otros} 
				\end{scope}
			\end{tikzpicture}
		\end{center}
	}
	
	En esta tabla se pueden consultar los resultados de todas las fuerzas políticas que obtuvieron representación parlamentaria.
	
	\noindent\makebox[\textwidth]{
		\console{
				\begin{tabular}{|l|r|r|r|}
				\hline
					\textbf{Lista electoral} & \textbf{Votos} & \textbf{Votos (\%)} & \textbf{Diputados} \\ \hline
					Partido Socialista Obrero Español\footnotemark & 11 026 163 & 42.59\% & 164 \\ \hline
					Partido Popular\footnotemark & 9 763 144 & 37.71\% & 148 \\ \hline
					Izquierda Unida\footnotemark & 1 284 081 & 4.96\% & 5 \\ \hline
					Conergència i Unió & 835 471 & 3.23\% & 10 \\ \hline
					Esquerra Republicana de Catalunya-Catalunya Sí & 652 196 & 2.52\% & 8 \\ \hline
					Euzko Alderdi Jeltzalea-Partido Nacionalista Vasco & 420 980 & 1.63\% & 7 \\ \hline
					Coalición Canaria\footnotemark & 235 22 & 0.91\% & 3 \\ \hline
					Bloque Nacionalista Gallego & 208 688 & 0.81\% & 2 \\ \hline
					Chunta Aragonesista & 94 252 & 0.36\% & 1 \\ \hline
					Eusko Alkartasuna & 80 905 & 0.30\% & 1 \\ \hline
					Nafarroa Bai & 61 045 & 0.24\% & 1 \\ 
				\hline	
				\end{tabular}
		}
	}
	
	\footnotetext[1]{Incluye a la Confederación de Los Verdes (LV) y Coalición Extremeña (PREx-CREx), y el apoyo de Unión Demócrata Ceutí (UDC).}
	\footnotetext[2]{Incluye a Unión del Pueblo Navarro (UPN), Unión del Pueblo Melillense (UPM), Unión Valenciana (UV) e Independientes de Fuerteventura.}
	\footnotetext[3]{Incluye a Iniciativa per Catalunya-Verds (ICV), Esquerra Unida i Alternativa (EUiA), Los Verdes de Aragón, Los Verdes de Canarias, Alternativa Ciudadana 25 de Mayo, Els Verds del País Valencià (EVPV), Socialistas Independientes de Extremadura (SIEX), Bloque por Asturies (BA), Izquierda Republicana (IR) en la Comunidad Valenciana, Partido Revolucionario de los Trabajadores-Izquierda Revolucionaria (PRT), Partido Obrero Revolucionario (POR), Corriente Roja y Espacio Alternativo, y el apoyo de Red Verde.}
	\footnotetext[4]{Incluye al Partido Nacionalista de Lanzarote (PNL).}

	\newpage
	
	\subsection{Elecciones del año 2015.}
	
	Las elecciones generales del año 2015 se celebraron el domingo 20 de diciembre. La aparición de dos nuevos partidos, Podemos y Ciudadanos, puso fin al bipartidismo imperante en España desde 1982. De las elecciones del 20D surgió un Congreso mucho más fragmentado, sin ninguna mayoría clara, que obligó a las fuerzas políticas a llegar a acuerdos para formar un Gobierno. La lista más votada fue la del Partido Popular, con 123 escaños (lejos de la mayoría absoluta, que está en 176 escaños), frente a los 90 del PSOE, los 69 de Podemos y los 40 de Ciudadanos.
	
	La composición del Congreso quedó como sigue:
	
	\console{
		\begin{center}
			\begin{tikzpicture}
				\begin{scope}[yscale=1,xscale=-1]
	 		  		\pie[pos={6,0}, sum=700, after number=, radius=4, color={pp, psoe, podemos, ciudadanos, gray }, text=legend]{123/PP, 90/PSOE, 69/Podemos, 40/Ciudadanos, 28/Otros} 
				\end{scope}
			\end{tikzpicture}
		\end{center}
	}
	
	En esta tabla se pueden consultar los resultados de todas las fuerzas políticas que obtuvieron representación parlamentaria.
	
	\console {	
		\noindent\makebox[\textwidth]{%
				\begin{tabular}{|l|r|r|r|}
				\hline
					\textbf{Lista electoral} & \textbf{Votos} & \textbf{Votos (\%)} & \textbf{Diputados} \\ \hline
					Partido Popular\footnotemark & 7 236 965 & 28.71\% & 121 \\ \hline
					Partido Socialista Obrero Español\footnotemark & 5 545 315 & 22.00\% & 90 \\ \hline
					Podemos \footnotemark & 5 212 711 & 20.66\% & 69 \\ \hline
					Ciudadanos-Partido de la Ciudadanía & 3 514 528 & 13.94\% & 40 \\ \hline
					Unidad Popular: IU, Unidad Popular en Común & 926 783 & 3.68\% & 2 \\ \hline
					Esquerra Republicana de Catalunya-Catalunya Sí & 601 782 & 2.39\% & 9 \\ \hline
					Democràcia i Llibertat & 567 253 & 2.25\% & 6 \\ \hline
					Euzko Alderdi Jeltzalea-Partido Nacionalista Vasco & 302 316 & 1.20\% & 6 \\ \hline
					Euskal Herria Bildu & 219 125 & 0.87\% & 2 \\ \hline
					Coalición Canaria-Partido Nacionalista Canario & 81 917 & 0.32\% & 1 \\
				\hline	
				\end{tabular}
		}
	}
	
	\footnotetext[5]{Incluye a Foro Asturias.}
	\footnotetext[6]{Incluye a Nueva Canarias.}
	\footnotetext[7]{Incluye a En Comú Podem, Compromís-Podemos-És el moment y En Marea.}
	
	
	
	\section{Simulaciones sobre los resultados electorales.}
	\subsection{Elecciones del año 2004.}
	\subsection{Elecciones del año 2015.}
	
\end{document}