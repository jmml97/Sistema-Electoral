\documentclass[11pt]{article}
\usepackage [utf8]{inputenc}
\usepackage [spanish]{babel}
\usepackage{multicol}
\usepackage{parskip}
\usepackage[margin=3cm]{geometry}
\usepackage{enumerate}
\usepackage{graphicx}
\usepackage{mathtools}
\usepackage{setspace}
\usepackage{pgfplots}
\usepackage{pgf-pie}
\usepackage{amsfonts}
\usepackage{xcolor}
\usepackage{colortbl}
\usepackage{framed}

\usepackage{lipsum}

%---- Fuentes

\usepackage{fontspec}
\setsansfont{Montserrat}[
  	Path        = ./fonts/ ,
	Extension		  = .ttf ,
	UprightFont		= *-Regular ,
	BoldFont		  = *-Bold]
\newfontfamily\headingfont{Montserrat}[
  	Path        = ./fonts/ ,
	Extension		  = .ttf ,
	UprightFont		= *-Regular ,
	BoldFont		  = *-Bold]
\setmainfont{Lora}[
	Path            = ./fonts/ ,
	Extension		    = .ttf ,
	UprightFont		  = *-Regular ,
	BoldFont		    = *-Bold ,
	ItalicFont		  = *-Italic ,
	BoldItalicFont	= *-BoldItalic]
\newfontfamily\inconsolata{Inconsolata}[
 	Path          = ./fonts/ ,
	Extension		  = .ttf ,
	UprightFont		= *-Regular ,
	BoldFont		  = *-Bold]

\newcommand\console[1]{{\inconsolata #1}}

%----- Encabezados

\usepackage{titlesec}
\usepackage{titling}

\renewcommand{\maketitlehooka}{\headingfont}

\titleformat*{\section}{\huge\bfseries\sffamily}
\titleformat*{\subsection}{\Large\bfseries\sffamily}
\titleformat*{\subsubsection}{\large\bfseries\sffamily}
\titleformat*{\paragraph}{\large\bfseries\sffamily}

%----- Colores

\definecolor{pp}{HTML}{00A3DF}
\definecolor{psoe}{HTML}{EE1C25}
\definecolor{podemos}{HTML}{672F6C}
\definecolor{ciudadanos}{HTML}{EB6109}
\definecolor{up}{HTML}{019839}

\begin{document}
	\title{
		Sistema electoral español\\
		\vspace*{1\baselineskip}
		\large 1ºDGIIM \\
		Métodos Numéricos 
		\vspace*{1\baselineskip}
	} 
	
	\date{31-03-16}
	\author{Antonio Coín Castro \and José María Martín Luque \and Miguel Lentisco Ballesteros}
	
	%\maketitle
	
	\fontsize{72}{60}{
		\headingfont{\raggedright{\textbf{Sistema}}} \\\\
		\headingfont{\raggedright{\textbf{Electoral}}} \\\\
		\headingfont{\raggedright{\textbf{Español}}} \\\\\\\\
	}
	
	\vfill
	
	\fontsize{20}{28}{
		\inconsolata{
			\raggedright{Antonio Coín Castro} \\
			\raggedright{Miguel Lentisco Ballesteros} \\
			\raggedright{José María Martín Luque} \\
		}
	}
	\thispagestyle{empty}
	\fontsize{11}{14}
	
	\newpage
	
	\tableofcontents
	
	\newpage
	
	\section*{Preámbulo.}
	\addcontentsline{toc}{section}{Preámbulo.}
	
	Todas las democracias tienen algo en común: \textit{la soberanía nacional reside en el pueblo}. Ahí se acaban las coincidencias: hay muchas formas de implementar la democracia. La más extendida es la democracia representativa: el pueblo elige a representantes que tomarán decisiones por ellos en los órganos de poder. España es un país con una democracia representativa. Los ciudadanos españoles eligen mediante el voto a sus representantes en el Congreso de los Diputados y el Senado, sedes del poder legislativo. Para transformar \textit{votos} en \textit{escaños} (así es como se denominan cada uno de los asientos que ocupan los representantes políticos en las cámaras parlamentarias) existen diversos métodos o sistemas. Explicar el funcionamiento de nuestro sistema es el tema que nos ocupa en este trabajo.
	
	El sistema electoral español para las elecciones al Congreso de los Diputados es un sistema sencillo de comprender, al menos comparado con el de otros países como Alemania. El objetivo de este documento no es sólo explicar el funcionamiento y en qué leyes se fundamenta el sistema electoral de nuestro país. También pretende mostrar los problemas que presenta, y exponer las principales críticas que recibe. 
	
	Para exponer mejor estas ideas este documento no sólo recoge información sobre nuestro sistema electoral. Ver el funcionamiento de otros sistemas ayuda a comparar ventajas y deficiencias. Por este motivo, en este texto también se comentan brevemente los sistemas vigentes en Alemania y Reino Unido (países que ya fueron mencionados anteriormente) así como otros métodos de reparto de escaños.
	
	Sin duda, apoyarse en datos reales es una de las mejores formas de presentar y explicar información. Este documento no se limita a explicar los fundamentos teóricos de los sistemas electorales analizados, sino que también recoge resultados reales de elecciones. Pero lo más interesante es utilizar estos datos y realizar simulaciones con ellos empleando distintos métodos. El uso de simulaciones permite apreciar de forma mucho más fácil las diferencias entre los sistemas electorales explicados teóricamente.
	
	Una vez finalizada la lectura de este escrito se espera que el lector haya adquirido el conocimiento suficiente como para comprender el funcionamiento de nuestro sistema electoral y cómo se compara con otros alternativos. De esta forma, podrá juzgar por sí mismo si se necesita o no un cambio en el sistema electoral español.
		
	\newpage
	
	\section{Sistema electoral español.}
	
	Actualmente el sistema electoral español está sustentado tanto en la Constitución de 1978 como en la Ley Orgánica 5/1985, de 19 de junio, del Régimen Electoral General. En este documento se recoge el sistema electoral para la elección de los diputados del Congreso. El Congreso de los Diputados es la sede del Poder Legislativo. Está formado por trescientos cincuenta Diputados, como recoge el artículo 162 de la Ley Electoral. La elección de los miembros de esta cámara se hace por sufragio universal entre los ciudadanos españoles mayores de 18 años.
	
	Los electores votan \textit{listas electorales}, una relación de aspirantes a obtener un escaño que se presentan bajo un mismo partido o coalición. Estas listas deben ser presentadas a nivel provincial. De esta forma, a cada provincia le corresponde un número concreto de diputados. Se dice que las provincias constituyen la circunscripción electoral. Por este motivo, los partidos presentan a sus candidatos en las distintas circunscripciones para intentar conseguir el mayor número de diputados posible. Cabe mencionar que estas listas electorales son \textit{cerradas}, pues las confecciona cada partido, y los electores no tienen posibilidad de votar a un representante, sino que votan a una lista al completo.
	
	La Constitución sienta las bases del sistema electoral, pero es una ley la que tendrá que regularlo por completo. Básicamente se limita a mencionar algunos aspectos básicos que debe cumplir dicha ley, como el número de diputados o la elección de éstos. \\
	
	\console{
		\begin{leftbar}
			\textbf{CONSTITUCIÓN ESPAÑOLA.} \\
			\textbf{Artículo 68.}
			
			\begin{enumerate}
			\item El Congreso se compone de un mínimo de 300 y un máximo de 400 Diputados, elegidos por sufragio universal, libre, igual, directo y secreto, en los términos que establezca la ley.
			\item La circunscripción electoral es la provincia. Las poblaciones de Ceuta y Melilla estarán representadas cada una de ellas por un Diputado. La ley distribuirá el número total de Diputados, asignando una representación mínima inicial a cada circunscripción y distribuyendo los demás en proporción a la población.
			\item La elección se verificará en cada circunscripción atendiendo a criterios de representación proporcional. \\
		\end{enumerate}	
		\end{leftbar}
	}
	
	La ley que cumple esta función es la Ley del Régimen Electoral General. En ella se acaban de detallar los procedimientos brevemente esbozados en la Constitución. Da cifras concretas del número de parlamentarios y especifica el método a seguir para traducir votos a escaños así como para calcular el número de escaños de cada circunscripción. Así, los artículos 161, 162 y 163 de esta ley detallan en profundidad los puntos 1, 2 y 3 del artículo 68 de la Constitución. \\
	
	\console{
	\begin{leftbar}
		\textbf{LEY DEL RÉGIMEN ELECTORAL GENERAL.} \\
		\textbf{Artículo 161.}
		
		\begin{enumerate}
			\item Para la elección de Diputados y Senadores, cada provincia constituirá una circunscripción electoral. Asimismo, las ciudades de Ceuta y Melilla serán consideradas, cada una de ellas, como circunscripciones electorales.
		\end{enumerate}
		
		\textbf{Artículo 162.}
		
		\begin{enumerate}
			\item El Congreso está formado por trescientos cincuenta Diputados.
			\item A cada provincia le corresponde un mínimo inicial de dos Diputados. Las poblaciones de Ceuta y Melilla están representadas cada una de ellas por un Diputado.
			\item Los doscientos cuarenta y ocho Diputados restantes se distribuyen entre las provincias en proporción a su población, conforme al siguiente procedimiento:
				\begin{enumerate}[a)]
					\item Se obtiene una cuota de reparto resultante de dividir por doscientos cuarenta y ocho la cifra total de la población de derecho de las provincias peninsulares e insulares.
					\item Se adjudican a cada provincia tantos Diputados como resulten, en números enteros, de dividir la población de derecho provincial por la cuota de reparto.
					\item Los Diputados restantes se distribuyen asignando uno a cada una de las provincias cuyo cociente, obtenido conforme al apartado anterior, tenga una fracción decimal mayor.
				\end{enumerate}
		\end{enumerate}
		
		\textbf{Artículo 163}.
	
		\begin{enumerate}
			\item La atribución de los escaños en función de los resultados del escrutinio se realiza conforme a las siguientes reglas:
			\begin{enumerate}[a)]
					\item No se tienen en cuenta aquellas candidaturas que no hubieran obtenido, al menos, el 3 por 100 de los votos válidos emitidos en la circunscripción.
					\item Se ordenan de mayor a menor, en una columna, las cifras de votos obtenidos por las restantes candidaturas.
					\item Se divide el número de votos obtenidos por cada candidatura por 1, 2, 3, etcétera, hasta un número igual al de escaños correspondientes a la circunscripción [...]. Los escaños se atribuyen a las candidaturas que obtengan los cocientes mayores en el cuadro, atendiendo a un orden decreciente.
				\end{enumerate}
		\end{enumerate}

	\end{leftbar}
	}
	
	\subsection	{Asignación de escaños a circunscripciones.}
	
	Como hemos visto en la Ley del Régimen Electoral General (en adelante \textit{LREG}) ilustrada anteriormente, las circunscripciones electorales en el sistema electoral español son 52: las 50 provincias españolas, Ceuta y Melilla. Tal y como se establece en la mencionada ley, el número de diputados que corresponden a cada circunscripción se asignan en cada convocatoria electoral, pues la población de derecho varía con el paso del tiempo.
	
	Así, no es posible saber a ciencia cierta cuántos escaños corresponden a cada circunscripción hasta que no se hace un recuento del censo electoral a fecha de la convocatoria de las elecciones. Veamos un ejemplo real para el cálculo de los diputados que correspondió elegir en las elecciones del pasado 20 de Diciembre de 2015 a los ciudadanos de la provincia de Almería.
	
	En primer lugar, tal y como garantiza la \textit{LREG}, a Almería le corresponden inicialmente dos escaños. Veamos los datos del censo provincial y estatal, que necesitaremos para hacer los cálculos:
	
	\begin{itemize}
	\item Población de derecho en la circunscripción de Almería: \textit{452.590}
	\item Población de derecho en el ámbito estatal: \textit{34.630.253}
	\end{itemize}
	
	Calculamos primero la cuota de reparto, resultante de dividir la población de derecho total por 248 (número de diputados restantes después de asignar dos a cada provincia, uno a Ceuta y otro a Melilla).

	$$\dfrac{34630253}{248} = 139638.1169\ \ \frac{personas}{esca\tilde{n}o}$$
	
	A continuación, dividimos por esa cifra la población con derecho a voto en la circunscripción de Almería:
	
	$$Esca\tilde{n}os_{Al} = \dfrac{452590}{139638.1169} = 3.24116371\ esca\tilde{n}os$$

  Ahora, puesto que no se contempla asignar una cantidad decimal de escaños, se toma la parte entera del número resultante, es decir, 3 escaños. Contando los dos anteriores, tenemos ya 5 escaños. Una vez realizado este proceso en todas las circunscripciones, se reparten los escaños sobrantes en función de una mayor parte decimal de la cuenta anterior, hasta que se completan los 350 escaños.
  
  Como consecuencia de este último recuento, Almería consiguió otro escaño. Así, para las elecciones generales del 20 de Diciembre de 2015, a la circunscripción de Almería le correspondían \textbf{6 escaños}.
  
	\subsection{Cómo se reparten los escaños.}
	
	El proceso descrito en la \textit{LREG} es un algoritmo de aplicación de la llamada \textit{Ley D'Hondt}, en honor a Victor D'Hondt, jurista belga que la concibió en 1874.
	
	La Ley D'Hondt o Sistema D'Hondt es un método de promedio mayor para asignar escaños en sistemas de representación proporcional por listas electorales. Los métodos de promedio mayor consisten en dividir los votos obtenidos por cada lista por cocientes sucesivos, y se asignan los escaños a los cocientes mayores que vayan resultando de las divisiones.
	
	Concretamente, este método de asignación de votos a escaños contempla la siguiente fórmula para el cálculo de los cocientes:
	
	$$i = 1,\ldots,N \Rightarrow C_i = \dfrac{V}{s + 1}\, ,$$
	
	donde $C_i$ es el cociente que resulta, en cada caso, de dividir los votos totales $V$ a cada lista por una cantidad $s + 1$, y $s$ es el número de escaños que la lista ya ha conseguido, inicialmente 0.
	
	El número de votos recibidos por cada lista se divide sucesivamente por cada uno de los divisores, desde 1 hasta el número total de escaños a repartir ($N$). La asignación de escaños se hace ordenando los cocientes de mayor a menor y asignando a cada uno un escaño, hasta que estos se agoten.
	
	Veamos ahora un ejemplo real de aplicación de la Ley D'Hondt, de nuevo fijándonos en los resultados electorales obtenidos en la provincia de Almería en las elecciones del 20 de Diciembre de 2015.
	
	Antes de nada, tenemos que tener en cuenta que se contempla en la \textit{LREG} que los partidos cuyo porcentaje de votos con respecto al total de votos válidos emitidos (incluyendo los votos en blanco) no llegue al 3\%, serán excluídos del conteo. Los resultados fueron los siguientes:
	
	\begin{itemize}
	\item Participación: \textit{311.059}
	\item Votos nulos: \textit{2.410}
	\item Votos en blanco: \textit{2.204} 
	\end{itemize}
	
	Así, el número mínimo de votos necesarios para optar a un escaño es:
	\begin{center}
	 \textit{(Votos totales - Votos nulos)$\cdot 0.03$} $= (311059 - 2410)\cdot 0.03 = 9259.47$ votos.
	\end{center}
	
	Una vez que se excluyeron aquellos partidos que no alcanzaron $9259$ votos, procedemos a la aplicación del algoritmo ya explicado, resultando en una tabla como la que sigue:
	
	\console{
		\begin{center}
			\begin{tabular}{|c|c|c|c|c|c|}
				\hline
		 		& \textbf{PP} & \textbf{PSOE} & \textbf{C's} & \textbf{PODEMOS} & \textbf{UP} \\
				\hline
				VOTOS & 117407 & 89022 & 44320 & 39482 & 10776 \\
				\hline
				ESCAÑO 1 &\cellcolor{gray!25} 117407 & 89022 & 44320 & 39482 & 10776 \\
				\hline
				ESCAÑO 2 & 58703 &\cellcolor{gray!25} 89022 & 44320 & 39482 & 10776 \\
				\hline
				ESCAÑO 3 &\cellcolor{gray!25} 58703 & 44511 & 44320 & 39482 & 10776 \\
				\hline
				ESCAÑO 4 & 39135 &\cellcolor{gray!25} 44511 & 44320 & 39482 & 10776 \\
				\hline
				ESCAÑO 5 & 39135 & 29674 &\cellcolor{gray!25} 44320 & 39482 & 10776 \\
				\hline
				ESCAÑO 6 & 39135 & 29674 & 22160 &\cellcolor{gray!25} 39482 & 10776 \\
				\hline
			\end{tabular}
		\end{center}
	}
	
	
\vspace{1em}
  Entonces, en la circunscripción de Almería correspondieron 2 escaños al PP (el primero y el tercero), dos escaños al PSOE (segundo y cuarto), y los dos últimos escaños los obtuvieron C's y PODEMOS, respectivamente.
  
  \textbf{Observaciones:}
  
  \begin{itemize}
  \item Es importante recalcar que en el método D'Hondt, los cocientes se redondean al \textbf{entero por defecto}.
  \item El precio de un escaño es en Almería fue de \textbf{39482} votos. Así, si la formación \textbf{UP} hubiera obtenido, por ejemplo, 25.000 votos más, seguiría sin obtener ningún escaño.
  \item El Partido Popular casi le arrebata el último escaño a PODEMOS, le faltaron \textbf{347} votos.
  \item El número total de votos a una lista no se tiene en cuenta, tan sólo los votos dentro de la circunscripción considerada en cada caso.
  \item Los partidos minoritarios ven cómo una gran cantidad de votos se ve "desperdiciada", pues superan con holgura los votos necesarios para obtener el primer escaño, pero no alcanzan el segundo.
	\end{itemize}
	
	
	\subsubsection{Votos blancos y votos nulos.}
	
	Merece especial atención el papel que juegan los votos en blanco y los votos nulos en las elecciones al Congreso.
	
	Los votos nulos simplemente se eliminan del conteo de votos de primera hora, es decir, votar nulo y no votar es lo mismo. Tan sólo se refleja el voto nulo en las estadísticas de participación.
	
	Sin embargo, votar en blanco sí tiene influencia en el resultado de las elecciones, y puede ser decisivo en algunos casos para determinar el acceso de un partido minoritario a un escaño en el Congreso. Un voto en blanco \textbf{es un voto válido}, y la Ley Electoral establece que aquellos partidos que, en cada circunscripción, no alcancen el 3\% del total de votos válidos quedarán excluídos del recuento de escaños para esa circunscripción.
	
	Hay cierta controversia con los votos en blanco, y en muchas ocasiones se escucha la frase $"$\textit{los votos en blanco favorecen a los partidos mayoritarios}$"$. Esta frase, aunque es cierta, es quizá un tanto exagerada. En teoría, podría darse una situación en que muchos votos en blanco en una circunscripción perjudicasen a un partido minoritario. Pensemos en una circunscripción grande, donde un partido tenga votos suficientes para conseguir un escaño, pero no alcance el 3\% de los votos válidos emitidos.
	
	Una vez que se han descartado los partidos que no llegan al 3\% de los votos, los votos en blanco no intervienen en la aplicación de la Ley D'Hondt.
	
	\subsection{Redondeo y distorsión.}
	
	Aunque el sistema electoral español se caracteriza por ser un sistema de representación proporcional, donde se persigue asignar los escaños a las listas de forma proporcional al número de votos recibidos, alcanzar la proporcionalidad exacta no es posible en general. En efecto, la imposibilidad de asignar un número decimal de escaños hace que inevitablemente se produzcan ciertos errores de redondeo.
	
	 Si retomamos el ejemplo anterior de la provincia de Almería, podemos hacer una comparativa de los escaños asignados a cada partido, y los escaños proporcionales que en teoría le corresponden a cada lista. Para ello, dividimos el número de votos de la lista entre el total de votos de los cinco partidos, y multiplicamos por los 6 escaños que se pretenden asignar:
	 
	 \console{
	 
	 \begin{center}
	 \begin{tabular}{l|c|c}
	 \textbf{Partido} & \textbf{Escaños asignados} & \textbf{Escaños proporcionales}\\
	 \hline
	 PP & 2 & 2.34\\
	 PSOE & 2 & 1.77\\
	 C's & 1 & 0.88\\
	 PODEMOS & 1 & 0.79\\
	 UP & 0 & 0.22
	 \end{tabular}
	 \end{center}
	 }
	 	 
	 Esta distorsión que se produce al aplicar el método \textit{D'Hondt} se puede cuantificar mediante la fórmula de \textit{Loosemore y Hanby}$^{[1]}$:
	 
	 \vspace{-0.5em}
	 $$D = \frac{1}{2} \sum_{i=1}^n |v_i - s_i|,$$
	 
	 \vspace{-0.5em}
	 la cual está acotada superiormente por
	 
	 \vspace{-0.5em}
	 $$D \le \frac{1}{2} \left[ (1 - v_w) + v_r(n-1) \right],$$
	 
	 \vspace{-0.5em}
	 donde:
	 
	 \enlargethispage{2\baselineskip}
	 
	  $\boldsymbol{n}$ es el número total de partidos.\\
  $\displaystyle \boldsymbol{v_{i}}\,$  es el porcentaje de voto del partido i-ésimo.\\
  $\displaystyle \boldsymbol{s_{i}}\,$  es el porcentaje de escaños del partido i-ésimo.\\
  $\displaystyle \boldsymbol{v_{w}}\,$  el umbral de votos con los cuales un partido obtendría todos los escaños de una circunscripción.\\
  $\displaystyle \boldsymbol{v_{r}}\,$  el umbral de votos mínimo a partir del cual un partido obtiene escaño en una circunscripción.
	 
	 Se puede demostrar que la Ley D'Hondt es la que más distorsión produce$^{[2]}$, en comparación con otras leyes o fórmulas de reparto proporcional de escaños similares.
	 
	 Podemos comprobar que la distorsión que se produjo en los resultados de la provincia de Almería, aplicando la fórmula, es de $D = 10.322$. Esta medida nos da una idea de la discrepancia entre los porcentajes de voto general de los electores a cada partido y los representantes paralmentarios obtenidos por cada uno de ellos.
	
	\section{Otros sistemas electorales.}
	
	En esta sección, comentaremos brevemente otros sistemas electorales, como los de Alemania o Reino Unido, los métodos que se usan en estos sistemas para asignar votos a escaños, y finalmente un ejemplo práctico para poder hacer una comparación de estos métodos con la Ley D'Hondt.
	
\paragraph{Reino Unido.}

En el Reino Unido cuentan con la Cámara de los Comunes, formada por 646 miembros que se eligen cada 5 años. Su sistema electoral es el uninominal mayoritario simple (comúnmente llamado $"$el ganador se lo lleva todo$"$), de manera que se divide en 646 circunscripciones o distritos electorales, todas del mismo tamaño. En cada circunscripción se elige un único escaño, que será el representante que más votos obtenga en ese distrito.

Algunos argumentos en contra de este sistema son el de que es poco proporcional (quizá uno de los menos proporcionales), o que provoca situaciones en las que partidos tienen muchos votos y pocos escaños (y viceversa). Los argumentos a favor pueden ser el hecho de que los representantes son directos (controlados por sus electores), y que forma un sistema estable, es decir, crea mayorías absolutas en general.

\paragraph{Alemania.}

En Alemania cuentan con el Bundestag, que no tiene un número fijo de diputados: se establece que al menos tengan 598 escaños, pero pueden ir añadiendose más para hacer el sistema más proporcional. Utilizan un sistema proporcional personalista a doble voto. El primer voto se realiza igual que en el Reino Unido, dividiéndose en 299 circunscripciones. Con el segundo voto se eligen 299 o más diputados de la siguiente forma: mediante listas cerradas se da el voto concreto al partido, y los escaños se asignan con un sistema llamado Sainte-Laguë, estableciéndose una barrera legal del 5\% para evitar la fragmentación del parlamento.

Este sistema tiene bastantes ventajas, como son una gran proporcionalidad que a la vez permite una buena representación regional, la posibilidad de elegir a los representantes de forma directa, y evita el bipartidismo favoreciendo la representación de los partidos pequeños. Un posible problema es su grado de complejidad, que puede provocar una bajada en la participación electoral.

\subsection{Método Sainte-Laguë}

Veamos en qué consiste el método Sainte-Laguë para asignar votos a escaños. Es un método de promedio mayor, al igual que la ley D'Hondt, y cuyos cocientes sucesivos se calculan mediante la siguiente expresión:

$$ i = 1,\ldots,N \Rightarrow C_i = \dfrac{V}{2s + 1}$$
	
	donde $C_i$ es el cociente que resulta, en cada caso, de dividir los votos totales $V$ a cada lista por una cantidad $2s + 1$, y $s$ es el número de escaños que la lista ya ha conseguido, inicialmente 0. Sin embargo, se suele usar una modificación de este sistema, donde cuando $s=0$ el cociente inicial es: $$ C_i = \frac{V}{1.4} $$

Retomemos el ejemplo de los resultados de las elecciones del 20 de Diciembre en la provincia de Almería, y veamos cómo se habrían repartido los escaños si hubiésemos usado el método Sainte-Laguë. Debemos notar que en este sistema, se utiliza el redondeo estándar (al entero más próximo).\\
\console{
\begin{center}
\begin{tabular}{|c|c|c|c|c|c|}
	\hline
	 & PP & PSOE & C's & PODEMOS & UP \\
	\hline
	VOTOS & 117407 & 89022 & 44320 & 39482 & 10776 \\
	\hline
	ESCAÑO 1 &\cellcolor{gray!25} 83862 & 63587 & 31657 & 28201 & 7697 \\
	\hline
	ESCAÑO 2 & 39136 &\cellcolor{gray!25} 63587 & 31657 & 28201 & 7697 \\
	\hline
	ESCAÑO 3 &\cellcolor{gray!25} 39136 & 29674 & 31657 & 28201 & 7697 \\
	\hline
	ESCAÑO 4 & 23481 & 29674 &\cellcolor{gray!25} 31657 & 28201 & 7697 \\
	\hline
	ESCAÑO 5 & 23481 &\cellcolor{gray!25} 29674 & 14773 & 28201 & 7697 \\
	\hline
	ESCAÑO 6 & 23481 & 17804 & 14773 &\cellcolor{gray!25} 28201 & 7697 \\
	\hline 
\end{tabular}
\end{center}
}

\vspace{0.5em}
Observamos que el reparto de escaños se mantiene igual que con la Ley D'Hondt. Sin embargo, ahora un escaño $"$cuesta$"$\hspace{0.2em} \textbf{28201}  votos. En esta configuración, el PP no ha estado tan cerca de arrebatarle un escaño a PODEMOS, y además el orden de obtención de los escaños cuarto y quinto se ha invertido.
	
	\newpage
	
	\section{Análisis de resultados electorales.}
	\subsection{Elecciones del año 2004.}
	
	Las elecciones generales del año 2004 tuvieron lugar el domingo 14 de marzo. Los españoles acudieron a las urnas a votar tan sólo tres días después de los atentados del 11M, en los que murieron 193 personas. A pesar de que todas las encuestas apuntaban a una victoria del PP, finalmente las elecciones fueron ganadas por el PSOE, lo que permitió a José Luis Rodríguez Zapatero convertirse en presidente del Gobierno.
	
	La composición del Congreso quedó como sigue:
	
	\console{
		\begin{center}
			\begin{tikzpicture}
				\begin{scope}[yscale=1,xscale=-1]
	 		  		\pie[pos={6,0}, sum=700, after number=, radius=4, color={psoe, pp, gray }, text=legend]{164/PSOE, 148/PP, 38/Otros} 
				\end{scope}
			\end{tikzpicture}
		\end{center}
	}
	
	En esta tabla se pueden consultar los resultados de todas las fuerzas políticas que obtuvieron representación parlamentaria.
	
	\noindent\makebox[\textwidth]{
		\console{
				\begin{tabular}{|l|r|r|r|}
				\hline
					\textbf{Lista electoral} & \textbf{Votos} & \textbf{Votos (\%)} & \textbf{Diputados} \\ \hline
					Partido Socialista Obrero Español\footnotemark & 11 026 163 & 42.59\% & 164 \\ \hline
					Partido Popular\footnotemark & 9 763 144 & 37.71\% & 148 \\ \hline
					Izquierda Unida\footnotemark & 1 284 081 & 4.96\% & 5 \\ \hline
					Conergència i Unió & 835 471 & 3.23\% & 10 \\ \hline
					Esquerra Republicana de Catalunya-Catalunya Sí & 652 196 & 2.52\% & 8 \\ \hline
					Euzko Alderdi Jeltzalea-Partido Nacionalista Vasco & 420 980 & 1.63\% & 7 \\ \hline
					Coalición Canaria\footnotemark & 235 22 & 0.91\% & 3 \\ \hline
					Bloque Nacionalista Gallego & 208 688 & 0.81\% & 2 \\ \hline
					Chunta Aragonesista & 94 252 & 0.36\% & 1 \\ \hline
					Eusko Alkartasuna & 80 905 & 0.30\% & 1 \\ \hline
					Nafarroa Bai & 61 045 & 0.24\% & 1 \\ 
				\hline	
				\end{tabular}
		}
	}
	
	\footnotetext[1]{Incluye a la Confederación de Los Verdes (LV) y Coalición Extremeña (PREx-CREx), y el apoyo de Unión Demócrata Ceutí (UDC).}
	\footnotetext[2]{Incluye a Unión del Pueblo Navarro (UPN), Unión del Pueblo Melillense (UPM), Unión Valenciana (UV) e Independientes de Fuerteventura.}
	\footnotetext[3]{Incluye a Iniciativa per Catalunya-Verds (ICV), Esquerra Unida i Alternativa (EUiA), Los Verdes de Aragón, Los Verdes de Canarias, Alternativa Ciudadana 25 de Mayo, Els Verds del País Valencià (EVPV), Socialistas Independientes de Extremadura (SIEX), Bloque por Asturies (BA), Izquierda Republicana (IR) en la Comunidad Valenciana, Partido Revolucionario de los Trabajadores-Izquierda Revolucionaria (PRT), Partido Obrero Revolucionario (POR), Corriente Roja y Espacio Alternativo, y el apoyo de Red Verde.}
	\footnotetext[4]{Incluye al Partido Nacionalista de Lanzarote (PNL).}

	\newpage
	
	\subsection{Elecciones del año 2015.}
	
	Las elecciones generales del año 2015 se celebraron el domingo 20 de diciembre. La aparición de dos nuevos partidos, Podemos y Ciudadanos, puso fin al bipartidismo imperante en España desde 1982. De las elecciones del 20D surgió un Congreso mucho más fragmentado, sin ninguna mayoría clara, que obligó a las fuerzas políticas a llegar a acuerdos para formar un Gobierno. La lista más votada fue la del Partido Popular, con 123 escaños (lejos de la mayoría absoluta, que está en 176 escaños), frente a los 90 del PSOE, los 69 de Podemos y los 40 de Ciudadanos.
	
	La composición del Congreso quedó como sigue:
	
	\console{
		\begin{center}
			\begin{tikzpicture}
				\begin{scope}[yscale=1,xscale=-1]
	 		  		\pie[pos={6,0}, sum=700, after number=, radius=4, color={pp, psoe, podemos, ciudadanos, gray }, text=legend]{123/PP, 90/PSOE, 69/Podemos, 40/Ciudadanos, 28/Otros} 
				\end{scope}
			\end{tikzpicture}
		\end{center}
	}
	
	En esta tabla se pueden consultar los resultados de todas las fuerzas políticas que obtuvieron representación parlamentaria.
	
	\console {	
		\noindent\makebox[\textwidth]{%
				\begin{tabular}{|l|r|r|r|}
				\hline
					\textbf{Lista electoral} & \textbf{Votos} & \textbf{Votos (\%)} & \textbf{Diputados} \\ \hline
					Partido Popular\footnotemark & 7 236 965 & 28.71\% & 123 \\ \hline
					Partido Socialista Obrero Español\footnotemark & 5 545 315 & 22.00\% & 90 \\ \hline
					Podemos \footnotemark & 5 212 711 & 20.66\% & 69 \\ \hline
					Ciudadanos-Partido de la Ciudadanía & 3 514 528 & 13.94\% & 40 \\ \hline
					Unidad Popular: IU, Unidad Popular en Común & 926 783 & 3.68\% & 2 \\ \hline
					Esquerra Republicana de Catalunya-Catalunya Sí & 601 782 & 2.39\% & 9 \\ \hline
					Democràcia i Llibertat & 567 253 & 2.25\% & 8 \\ \hline
					Euzko Alderdi Jeltzalea-Partido Nacionalista Vasco & 302 316 & 1.20\% & 6 \\ \hline
					Euskal Herria Bildu & 219 125 & 0.87\% & 2 \\ \hline
					Coalición Canaria-Partido Nacionalista Canario & 81 917 & 0.32\% & 1 \\
				\hline	
				\end{tabular}
		}
	}
	
	\footnotetext[5]{Incluye a Foro Asturias.}
	\footnotetext[6]{Incluye a Nueva Canarias.}
	\footnotetext[7]{Incluye a En Comú Podem, Compromís-Podemos-És el moment y En Marea.}

	
		\enlargethispage{2\baselineskip}
	Como podemos observar, existen algunas incongruencias entre el porcentaje de votos y el porcentaje de diputados obtenidos por las diversas formaciones.
	
	Por ejemplo, el PSOE tiene 21 diputados más que PODEMOS, contando tan sólo con unos 330.000 votos más. Sin embargo, el caso más reseñable es el de IU-UP, que con más de 920.000 ha obtenido tan sólo dos escaños. Podríamos decir que un escaño de IU $"$ha costado$"$  460.000 votos, mientras que por ejemplo un escaño del PP $"$ha costado$"$ unos 60.000 votos.\\
	
	Podemos concluir, a la vista también de los resultados de las elecciones de 2004, que la Ley D'Hondt, aplicada junto con circunscripciones provinciales y el resto de las consideraciones de la Ley Electoral actual, suscita un sistema mayoritario, donde los grandes partidos son beneficiados de manera notable. Esto evita la fragmentación del Congreso y busca una gobernabilidad mayor, pero a cambio el sistema pierde bastante proporcionalidad.
	
	
	\section{Simulaciones sobre los resultados electorales.}
	\subsection{Elecciones del año 2015.}
	
	Proporcionamos una página web donde se pueden realizar simulaciones de distinto tipo sobre las elecciones al Congreso de los Diputados del pasado 20 de Diciembre de 2015. En concreto, se puede manipular el número de diputados, el método de asignación de votos a escaños, la barrera electoral, la configuración de las circunscripciones, e incluso las alianzas entre algunos partidos.
	
	Se puede acceder a esta web a través de la siguiente dirección:\\
	
	\vspace{-1em}
	\console{\textit{https://jmml97.github.io/MiCongreso/}}
	
	\subsubsection{Circunscripción única con corte en el 3\%.}
	
	Hacemos una simulación con circunscripción única. Esto quiere decir que se hace un cómputo a nivel estatal para calcular los escaños, en este caso por la Ley D'Hondt. Como en la legislación actual, no se incluyen aquellos partidos que no alcancen el 3\% de los votos, solo que aquí es a nivel de todo el Estado. Por este motivo, los distintos partidos nacionalistas se quedan fuera del parlamento.
	
	\console{
		\begin{center}
			\begin{tikzpicture}
				\begin{scope}[yscale=1,xscale=-1]
	 		  		\pie[pos={6,0}, sum=700, after number=, radius=4, color={pp, psoe, podemos, ciudadanos, up }, text=legend]{113/PP, 87/PSOE, 81/Podemos, 55/Ciudadanos, 14/IU UPeC} 
				\end{scope}
			\end{tikzpicture}
		\end{center}
	}
	
	Los partidos que habrían obtenido representación parlamentaria serían: 
	
	\console {	
		\noindent\makebox[\textwidth]{%
				\begin{tabular}{|l|r|r|r|}
				\hline
					\textbf{Lista electoral} & \textbf{Dip. simulación} & \textbf{Diputados 20D} & \textbf{Diferencia}\\ \hline
					Partido Popular & 113 & 123 & -10  \\ \hline
					Partido Socialista Obrero Español & 87 & 90 & -3 \\ \hline
					Podemos y Confluencias & 81 & 69 & +12 \\ \hline
					Ciudadanos-Partido de la Ciudadanía & 55 & 40 & +15 \\ \hline
					Unidad Popular: IU, Unidad Popular en Común & 14 & 2 & +12 \\
				\hline	
				\end{tabular}
		}
	}
	
	El resto de partidos no obtiene ningún escaño. Observamos claramente cómo este sistema favorece a los partidos minoritarios, y proporciona un Congreso muy poco fragmentado. Tan sólo cinco partidos han obtenido representación en el Congreso, dejando fuera al resto de partidos nacionalistas y minoritarios.
	
	\subsubsection{Circunscripción única con corte en el 1\%.}
	
	Repetimos la simulación anterior, pero esta vez el corte para descartar los partidos que obtienen escaño está en el 1\%. Así, varios partidos nacionalistas obtienen escaño, pero no es así con EH Bildu y Coalición Canaria, que al no llegar al 1\% de los votos, quedan descartados. El hemiciclo queda como sigue:
	
	\console{
		\begin{center}
			\begin{tikzpicture}
				\begin{scope}[yscale=1,xscale=-1]
	 		  		\pie[pos={6,0}, sum=700, after number=, radius=4, color={pp, psoe, podemos, ciudadanos, up, gray }, text=legend]{107/PP, 82/PSOE, 77/Podemos, 51/Ciudadanos, 13/IU UPeC, 20/Otros} 
				\end{scope}
			\end{tikzpicture}
		\end{center}
	}
	
	Los partidos que habrían obtenido representación parlamentaria serían: 
	
	\console {	
		\noindent\makebox[\textwidth]{%
				\begin{tabular}{|l|r|r|r|}
				\hline
					\textbf{Lista electoral} & \textbf{Dip. simulación} & \textbf{Diputados 20D} & \textbf{Diferencia}\\ \hline
					Partido Popular & 107 & 123 & -16  \\ \hline
					Partido Socialista Obrero Español & 82 & 90 & -8 \\ \hline
					Podemos y Confluencias & 77 & 69 & +8 \\ \hline
					Ciudadanos-Partido de la Ciudadanía & 51 & 40 & +11 \\ \hline
					Unidad Popular: IU, Unidad Popular en Común & 13 & 2 & +11 \\ \hline
					Esquerra Republicana de Catalunya-Catalunya Sí & 8 & 9 & -1 \\ \hline
					Democràcia i Llibertat & 8 & 6 & +2 \\ \hline
					Euzko Alderdi Jeltzalea-Partido Nacionalista Vasco & 6 & 6 & 0 \\
				\hline	
				\end{tabular}
		}
	}
	
	De nuevo vemos como este método favorece a los partidos minoritarios y perjudica a los grandes. Sin embargo, el problema de la poca diversidad en el Congreso quedaría subsanado, pues habría otros tres partidos (nacionalistas) que accederían a ocupar algún escaño.
	
	\subsubsection{Unidos Podemos.}
	
	Para comprobar cómo nuestro sistema premia las coaliciones, en el sentido de que son más eficientes para obtener escaños al aglutinar el voto, vamos a hacer una simulación en la que se suman los votos de Izquierda Unida y Podemos (que el 26J se presentarán en coalición). La circunscripción será provincial y el sistema de reparto, D'Hondt.
	
	\console{
		\begin{center}
			\begin{tikzpicture}
				\begin{scope}[yscale=1,xscale=-1]
	 		  		\pie[pos={6,0}, sum=700, after number=, radius=4, color={pp, psoe, podemos, ciudadanos, gray }, text=legend]{116/PP, 88/PSOE, 85/Unidos Podemos, 36/Ciudadanos, 25/Otros} 
				\end{scope}
			\end{tikzpicture}
		\end{center}
	}
	
	Los partidos que habrían obtenido representación parlamentaria serían:
	
	\console {	
		\noindent\makebox[\textwidth]{%
				\begin{tabular}{|l|r|r|r|}
				\hline
					\textbf{Lista electoral} & \textbf{Dip. simulación} & \textbf{Diputados 20D} & \textbf{Diferencia}\\ \hline
					Partido Popular\footnotemark & 116 & 123 & -7 \\ \hline
					Partido Socialista Obrero Español\footnotemark & 88 & 90 & -2 \\ \hline
					Unidos Podemos \footnotemark & 85 & 71 & +14 \\ \hline
					Ciudadanos-Partido de la Ciudadanía & 36 & 40 & -4 \\ \hline
					Esquerra Republicana de Catalunya-Catalunya Sí & 9 & 9 & 0\\ \hline
					Democràcia i Llibertat & 8 & 8 & 0 \\ \hline
					Euzko Alderdi Jeltzalea-Partido Nacionalista Vasco & 5 & 6 & -1\\ \hline
					Euskal Herria Bildu & 2 & 2 & 0 \\ \hline
					Coalición Canaria-Partido Nacionalista Canario & 1 & 1 & 0 \\
				\hline	
				\end{tabular}
		}
	}
	
	Se puede observar claramente lo rentable que hubiese sido la coalición. Unidad Popular y Podemos obtuvieron en total 71 escaños el 20D. En esta simulación podemos observar que con los mismos votos habrían obtenido 14 diputados más. La mayoría de los escaños se los araña al PP, que pierde 7. También al PSOE (-2), Ciudadanos (-4) y PNV (-1).
	
	También podemos ver una de las principales paradojas de nuestro sistema electoral. Debido a la circunscripción provincial, Unidos Podemos habría obtenido más votos que el PSOE (24.34\% frente a 22.00\%) y sin embargo, conseguiría 3 escaños menos.

	\section{Conclusiones.}
	
	\subsection{Críticas al sistema electoral actual.}
	
	\paragraph{Falta de proporcionalidad.}
	
	La Constitución dice en su artículo número 68 que "la elección se verificará \textbf{en cada circunscripción} atendiendo a criterios de representación proporcional". Es decir, nuestro sistema electoral \textbf{no es proporcional} en el conjunto del Estado, pero sí que lo es en la \textbf{provincia}. O lo que es lo mismo, la propia Constitución no asegura que la distribución de escaños a nivel estatal se corresponda de forma proporcional al cómputo total de los votos. Buena prueba de ello es otra de las paradojas de nuestro sistema electoral: un partido podría conseguir un porcentaje determinado de votos, y sus escaños correspondientes en unas elecciones, y 4 años después, obtener un porcentaje de votos mayor, y menos escaños en otras elecciones.
	
	\paragraph{Circunscripción provincial.}
	
	Se suele culpar a la Ley D'Hondt de la falta de proporcionalidad en los resultados Estatales. Sin embargo, la principal causante de este problema (que como se ha comentado en el punto anterior viene implícitamente contemplado en la Constitución) es la circunscripción provincial. Muchos politólogos argumentan que realmente en España hay tres elecciones, con tres sistemas \textit{distintos} atendiendo al número de diputados a elegir por circunscripción. Las circunscripciones que reparten de 1 a 6 escaños son sistemas mayoritarios, el partido que más votos recibe puede acaparar gran parte de los escaños (los datos de todas elecciones nos dan una barrera media efectiva del 15,2\%). Las circunscripciones que reparten entre 6 y 9 escaños se asemejan a un sistema intermedio entre mayoritario y proporcional (la barrera efectiva media aquí es del 9,8\%). Finalmente, las provincias que reparten más de 10 escaños podemos decir que son puramente proporcionales y la barrera efectiva media se acerca a la impuesta por la Ley Electoral, es del 3,1\%.
	
	\paragraph{Mínimo de escaños por provincia.}
	
	Según la Ley Electoral actual, hay un mínimo de dos escaños asignados a cada provincia. Esto hace que provincias con pocos habitantes se vean sobre-representadas, pues por ejemplo, una provincia con aproximadamente los mismos habitantes que Ceuta o Melilla (e incluso menos) tendría, al menos, un escaño más que cualquiera de las dos ciudades autónomas. 
	
	\paragraph{Número de diputados en el Congreso.}
	
	La población en España ha ido aumentando desde que se estableciera la Ley Electoral en 1985, y sin embargo el número de diputados ha permanecido igual desde entonces. Este hecho contribuye a un mayor desequilibrio en la proporción entre votos y escaños. 
	
	\subsection{Propuestas de mejora.}
	
	Finalmente, proponemos algunos cambios en el sistema electoral actual, que a nuestro juicio contribuirían a mejorarlo, aumentando la proporcionalidad y evitando el favorecimiento a los partidos mayoritarios, sin que por ello disminuya la gobernabilidad.
	
	\paragraph{Aumentar el número de diputados.}
	
	El hecho de aumentar los 350 diputados a 400 haría que aumentase la proporcionalidad en el reparto de escaños, pues la Ley D'Hondt funciona mejor cuanto mayor es el número de escaños a repartir.
	
	\paragraph{Cambiar el mínimo de diputados por provincia.}
	
	Una opción para aumentar la proporcionalidad sería reducir a 1 el número mínimo de escaños garantizado para cada provincia, garantizando así que los escaños restantes se reparten de forma proporcional a la población de cada una de las provincias.
	
	\paragraph{Cambiar las circunscripciones.}
	
	Este es sin duda el cambio más importante, que supondría un punto de inflexión en el sistema electoral español. El hecho de tener tantas circunscripciones hace que en algunos lugares se sobre-represente el voto del electorado, y en otras se infra-representa. Una posible solución sería la circunscripción única, aunque como ya hemos comprobado en la simulación se perdería la representación territorial (puede argumentarse que esa función la cumple el Senado). Otra opción, quizás más asequible y que busca un punto de equilibrio, sería confeccionar las circunscripciones por Comunidades Autónomas o, en su defecto, fusionar las circunscripciones ya existentes en función de su peso demográfico.
	
		\paragraph{Listas abiertas.}
	
	Las listas abiertas o desbloqueadas propician el acercamiento los candidatos a los electores, ya que estos deben hacer campaña individualmente para intentar obtener mejores resultados que el resto de candidatos (y obtener el escaño).
	
	
	\section*{Bibliografía.}
	\addcontentsline{toc}{section}{Bibliografía.}

	[1] Loosemore, J. \& Hanby, V. (1971): "The theoretical limits of maximum distortion: some analytic expressions for electoral systems". British Journal of Political Science, 467-477.\\

	[2] Laakso, M. (1979): "The Maximum Distortion and the Problem of the First Divisor of Different P.R. Systems". Research Note.\\
	
	
\end{document}